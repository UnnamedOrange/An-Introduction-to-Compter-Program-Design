% Licensed under the Creative Commons Attribution Share Alike 4.0 International.
% See the LICENSE file in the repository root for full license text.

\begin{problemset}
	\item 线性筛。阅读下面的代码。

	\begin{lstlisting}[language=c, moreemph={[2]sieve}]
#include <stdio.h>

int prime[664580];
int isntPrime[(int)1e7 + 1];
void sieve()
{
	prime[0] = 0;
	for (int i = 2; i <= (int)1e7; i++)
	{
		if (!isntPrime[i])
			prime[++prime[0]] = i;
		for (int j = 1, s = i * prime[j];
			j <= prime[0] && s <= (int)1e7; j++, s = i * prime[j])
		{
			isntPrime[s] = 1;
			if (!(i % prime[j]))
				break;
		}
	}
}

int main()
{
	sieve();
	for (int i = 1; i <= 25; i++)
		printf("%d ", prime[i]);
}
	\end{lstlisting}

	\begin{enumerate}
		\item 这个程序会输出一些整数。这些整数的含义是什么?

		提示:你可以查阅变量名的中文翻译,也可以运行程序。

		\item 仔细阅读 \lstinline[language=c, moreemph={[2]sieve}]{sieve} 函数(第 7-19 行),回答:
		\begin{enumerate}
			\item \lstinline{isntPrime} 数组的含义是什么?
			\item \lstinline{isntPrime} 数组的大小需要在一千万的基础上加一的原因是什么?
			\item \lstinline[language=c, moreemph={[2]sieve}]{sieve} 函数的作用是什么?
			\item \lstinline{prime[0]} 的含义是什么?
			\item \lstinline{prime} 数组的大小 \lstinline{664580} 的含义是什么?
		\end{enumerate}

		\item 已知该程序执行的指令数可以被第 15 行代码执行次数的有限倍数控制。想办法估计该程序的时间复杂度。规定自变量为 $n$,该代码中 $n = 10^7$。

		\begin{lstlisting}[language=c, firstnumber=15]
			isntPrime[s] = 1;
		\end{lstlisting}

		提示:记录这句代码被执行的次数。
	\end{enumerate}

	\item 冒泡排序。阅读下面的代码。

	\begin{lstlisting}[language=c, moreemph={[2]bubble_sort}]
#include <stdio.h>

void bubble_sort(int* a, int n)
{
	for (int i = 0; i < n - 1; i++)
		for (int j = 0; j < n - 1; j++)
			if (a[j] > a[j + 1])
			{
				int t = a[j];
				a[j] = a[j + 1];
				a[j + 1] = t;
			}
}

int a[10005];

int main()
{
	int n;
	scanf("%d", &n);
	for (int i = 0; i < n; i++)
		scanf("%d", a + i);
	bubble_sort(a, n);
	for (int i = 0; i < n; i++)
		printf("%d ", a[i]);
}
	\end{lstlisting}

	函数 \lstinline[language=c, moreemph={[2]bubble_sort}]{bubble_sort} 的作用是将数组 \lstinline{a} 从小到大排序。回答下面的问题。

	\begin{enumerate}
		\item 对该程序输入以下内容。

		\begin{lstlisting}[numbers=none]
6
1 1 4 5 1 4
		\end{lstlisting}

		该程序会输出什么?

		\item 数组 \lstinline{a} 为什么要定义为全局变量?

		\item 仔细阅读 \lstinline[language=c, moreemph={[2]bubble_sort}]{bubble_sort} 函数(第 5-12 行),回答:
		\begin{enumerate}
			\item 该程序的时间复杂度是多少?
			\item 这个函数(即冒泡排序)的思路是什么?
			\item 第 6 行中 \lstinline{n} 额外减去 \lstinline{1} 的原因是什么?
			\item 第 5 行中 \lstinline{n} 额外减去 \lstinline{1} 的原因是什么?
			\item 第 6 行的循环条件(\lstinline[language=c]{j < n - 1})可以稍作修改,使得循环次数更少,但程序仍然正确。可以改成什么?修改后,程序的时间复杂度是多少?
		\end{enumerate}

		提示:弄清楚冒泡排序的思路才能轻松地回答后面的问题。

		\item 补全下面的程序,实现对英文单词按字典序从小到大进行排序(假设输入的单词都是小写,没有空格,单词的长度不超过 79)。

		\begin{lstlisting}[language=c, moreemph={[2]bubble_sort_for_string}]
#include <stdio.h>
#include <string.h>

void bubble_sort_for_string(char(*strings)[80], int n)
{
	// 补全这里
}

char strings[20][80];

int main()
{
	int n;
	scanf("%d", &n);
	for (int i = 0; i < n; i++)
		scanf("%s", strings + i);
	bubble_sort_for_string(strings, n);
	for (int i = 0; i < n; i++)
		printf("%s\n", strings[i]);
}
		\end{lstlisting}

		样例输入:
		\begin{lstlisting}[numbers=none]
4
xyz
abandon
abc
zone
		\end{lstlisting}

		样例输出:
		\begin{lstlisting}[numbers=none]
abandon
abc
xyz
zone
		\end{lstlisting}

		补全后的程序的时间复杂度是多少?规定自变量为字符串数量 $n$ 和字符串最大长度 $m$(本小问中 $m$ 取固定值 80,但分析时间复杂度时规定字符串最大长度是一个变量)。

		提示:\lstinline[language=c, moreemph={[2]bubble_sort_for_string}]{bubble_sort_for_string} 函数的第一个参数 \lstinline{strings} 是一个指向长度为 80 的字符数组的指针,可以像 \lstinline[language=c]{int*} 使用。
	\end{enumerate}

	\item 交错数组。阅读下面的代码。

	\begin{lstlisting}[language=c, moreemph={[2]create, destroy, malloc, free}]
#include <stdio.h>
#include <string.h>
#include <stdlib.h>

int** create(int n)
{
	int** root = (int**)malloc(sizeof(int*) * n);
	for (int i = 0; i < n; i++)
		root[i] = (int*)malloc(sizeof(int) * (i + 1));
	return root;
}

void destroy(int** root, int n)
{
	for (int i = 0; i < n; i++)
		free(root[i]);
	free(root);
}

int main()
{
	int n;
	scanf("%d", &n);

	{
		int** C = create(n);
		C[0][0] = C[1][0] = C[1][1] = 1;
		for (int i = 2; i < n; i++)
		{
			C[i][0] = C[i][i] = 1;
			for (int j = 1; j < i; j++)
				C[i][j] = C[i - 1][j - 1] + C[i - 1][j];
		}

		for (int i = 0; i < n; i++)
		{
			for (int j = 0; j < i; j++)
				printf("%d ", C[i][j]);
			printf("\n");
		}
		destroy(C, n);
	}
}
	\end{lstlisting}

	回答下面的问题。

	\begin{enumerate}
		\item 假设解决方案平台为“x64”,求以下三个表达式的值。

		\begin{lstlisting}[language=c, numbers=none]
// int** C;
sizeof(C);
sizeof(C[0]);
sizeof(C[0][0]);
		\end{lstlisting}

		\item 仔细阅读 \lstinline[language=c, moreemph={[2]create}]{create} 函数(第 7-10 行)。假设解决方案平台为“x64”,以 $n$ 为自变量,该函数动态申请了多少字节的堆空间?

		\item 假设输入 $n \le 5000$,使用一个全局的二维数组代替动态内存分配,则这个数组的大小至少是多少?至少占用多少字节的空间?

		\item 该程序的作用是输出著名的杨辉三角,\lstinline{C} 表示组合数。

		\begin{enumerate}
			\item 该程序的时间复杂度是多少(以 $n$ 为自变量,使用大 $O$ 记号表示,尽可能接近上确界)?
			\item 如果要求只输出一个组合数 $\mathrm C_n^m$,则对于该程序而言只需要修改程序的输出部分(第 35-40 行)。但该程序的时间复杂度太高,所以请设计一个时间复杂度为 $O(n)$ 的算法解决只输出一个组合数的问题,并使用 C 语言编写一个完整程序(\textbf{不考虑整数溢出})。

			提示:组合数的通项公式为:
			$$
			\mathrm C_n^m = \dfrac{n!}{m! (n - m)!}
			$$
		\end{enumerate}
	\end{enumerate}

	\item \lstinline[language=c, moreemph={[2]memset}]{memset} 函数。\lstinline[language=c, moreemph={[2]memset}]{memset} 函数的作用是将一片内存区域\textbf{以字节为单位}进行填充,使用时首先需要包含 \lstinline{string.h} 头文件。回答下面的问题。

	\begin{enumerate}
		\item 有人写出了以下代码。

		\begin{lstlisting}[language=c, numbers=none, moreemph={[2]memset}]
	int a;
	memset(&a, 0xcc, sizeof(a));
		\end{lstlisting}

		\lstinline[language=c, moreemph={[2]memset}]{memset} 后,变量 \lstinline{a} 的值变成了 \lstinline{0xcccccccc}。解释 \lstinline[language=c, moreemph={[2]memset}]{memset} 的三个参数的含义。

		提示:你可以通过多种途径得到它们的含义。

		\item 有人写出了以下代码。

		\begin{lstlisting}[language=c, moreemph={[2]memset, malloc, free}]
#include <stdio.h>
#include <stdlib.h>
#include <string.h>

int main()
{
	int* p = malloc(sizeof(int) * 114514);
	memset(p, 0, sizeof(p));
	for (int i = 0; i < 114514; i++)
		printf("%d\n", p[i]); // 期望输出 0。
	free(p);
}
		\end{lstlisting}

		这段代码错误地使用了 \lstinline[language=c, moreemph={[2]memset}]{memset} 函数,请修正这个问题。已知 \lstinline[language=c, moreemph={[2]malloc}]{malloc} 函数为我们分配的空间默认是没有被初始化的。
	\end{enumerate}

	\item 小端字序与大端字序。阅读下面的代码。

	\begin{lstlisting}[language=c, moreemph={[1]uint8_t, uint16_t, uint32_t, uint64_t, MyUInt64}, moreemph={[2]putchar}]
#include <stdio.h>
#include <stdint.h>

typedef union
{
	uint8_t bytes[8];
	uint16_t words[4];
	uint32_t dwords[2];
	uint64_t qword;
} MyUInt64;

int main()
{
	MyUInt64 n;
	scanf("%llu", &n);
	for (int i = 0; i < 8; i++)
		printf("%02x ", n.bytes[i]);
	putchar('\n');
	for (int i = 0; i < 4; i++)
		printf("%04x ", n.words[i]);
	putchar('\n');
	for (int i = 0; i < 2; i++)
		printf("%08x ", n.dwords[i]);
	putchar('\n');
	printf("%016llx\n", n.qword);
}
	\end{lstlisting}

	回答下面的问题。

	\begin{enumerate}
		\item 该程序中的类型 \lstinline[language=c, moreemph={[1]MyUInt64}]{MyUInt64} 占用多少字节的空间?
		\item 运行该程序,输入 \lstinline{114514114514114514},记录程序的输出。
		\item 根据上一小问的运行结果猜测格式化说明符 \lstinline[language=c]{"%02x"} 的含义。
		\item 对于一个占用多个字节的整数,如果整数的低位对应的地址比整数的高位对应的地址要小,则称这是\textbf{小端字序(little endian)};反之称为\textbf{大端字序(big endian)}。根据第二小问的运行结果判断你的计算机编译出的 C 语言程序采用的是小端字序还是大端字序,并说明理由。
	\end{enumerate}

	\item 位段与内存对齐。

	\begin{enumerate}
		\item 位段。C 语言的结构体允许定义长度为若干比特的整数,方法是在整数变量的定义之后加上一个冒号(\lstinline{:}),后面再接上比特位宽(不能超过对应整数类型的比特数)。整数是否有符号继承自原类型。

		例如,可以写出以下代码。

		\begin{lstlisting}[language=c]
#include <stdio.h>

int main()
{
	struct
	{
		unsigned on_off : 1; // 开关,取值为无符号的 0 和 1。
		signed a_number : 7; // 7 位有符号整数。
	} t;
	t.on_off = 1;
	for (int i = -128; i < 127; i++)
		printf("%d ", t.a_number = i);
}
		\end{lstlisting}

		\lstinline{t.a_number} 表示的整数的范围是多少?

		\item 内存对齐。运行以下程序。
		\begin{lstlisting}[language=c]
#include <stdio.h>

int main()
{
	struct
	{
		unsigned on_off : 1; // 开关,取值为无符号的 0 和 1。
		signed a_number : 7; // 7 位有符号整数。
	} t;
	printf("%d\n", (int)sizeof(t));
}
		\end{lstlisting}

		程序输出了什么?尝试解释输出结果。
	\end{enumerate}

	\item 在线评测平台题单。

	\begin{enumerate}
		\item 在在线评测平台(OJ)上完成下面的所有题目,得到 Accepted(AC)的反馈。其中的“自选”表示你自己感兴趣的其他题目,也必须完成。自选的题目不限制 OJ 平台。
		\item 从下面的题目中选择 2 道(包括自选)撰写题解。题解应当包含题意解析、你的分析思路、你遇到的问题、你的标准解法。
	\end{enumerate}

	\begin{table}[H]
		\centering
		\begin{tabular}{|c|c|c|}
			\hline
			题号 & 题目名称 & 备注
			\\\hline
			\href{https://www.luogu.com.cn/problem/P1428}{洛谷 P1428} & 小鱼比可爱 &
			\\\hline
			\href{https://www.luogu.com.cn/problem/P5728}{洛谷 P5728} & 旗鼓相当的对手 &
			\\\hline
			\href{https://www.luogu.com.cn/problem/P2615}{洛谷 P2615} & 神奇的幻方 &
			\\\hline
			\href{https://www.luogu.com.cn/problem/P5730}{洛谷 P5730} & 显示屏 &
			\\\hline
			\href{https://www.luogu.com.cn/problem/P1614}{洛谷 P1614} & 爱与愁的心痛 & 分析时间复杂度
			\\\hline
			\href{https://www.luogu.com.cn/problem/P1789}{洛谷 P1789} & 插火把 &
			\\\hline
			\href{https://www.luogu.com.cn/problem/P1765}{洛谷 P1765} & 手机 & 可以使用字符串字面量初始化作为常量的字符数组
			\\\hline
			\href{https://www.luogu.com.cn/problem/P5738}{洛谷 P5738} & 歌唱比赛 &
			\\\hline
			\href{https://www.luogu.com.cn/problem/P5740}{洛谷 P5740} & 最厉害的学生 &
			\\\hline
			\href{https://www.luogu.com.cn/problem/P1304}{洛谷 P1304} & 哥德巴赫猜想 & 提前计算出某个数是否为质数
			\\\hline
			自选 & &
			\\\hline
			自选 & &
			\\\hline
			自选 & &
			\\\hline
		\end{tabular}
	\end{table}

\end{problemset}