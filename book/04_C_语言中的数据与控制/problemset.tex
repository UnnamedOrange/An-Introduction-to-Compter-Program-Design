% Licensed under the Creative Commons Attribution Share Alike 4.0 International.
% See the LICENSE file in the repository root for full license text.

\begin{problemset}
	\item lowbit。阅读下面的代码。

	\begin{lstlisting}[language=c]
#include <stdio.h>

int main()
{
	signed char x;
	{
		int input;
		scanf("%d", &input);
		x = input;
	}
	printf("%d\n", (signed char)(x & (-x)));
}
	\end{lstlisting}

	其中,\lstinline{&} 表示按位与,含义是将两个二进制整数的每一位依次进行“与运算”,“与运算”的真值表为:
	\begin{table}[H]
		\centering
		\begin{tabular}{c|c|c|}
			& 0 & 1
			\\\hline
			0 & 0 & 0
			\\\hline
			1 & 0 & 1
			\\\hline
		\end{tabular}
	\end{table}

	即均为 1 才为 1,否则为 0。

	\begin{enumerate}
		\item 从第 8 行可以看到,该程序运行时需要输入一个整数。若第 9 行对 \lstinline{x} 赋值后 \lstinline{x} 表示的整数与输入的整数相等,则输入的整数的范围应当是?

		\item \lstinline[language=c]{(signed char)(...)} 表示将括号内的值强制转换为 \lstinline[language=c]{signed char} 类型,转换规则是将括号内的整数直接截断为最低的 8 个二进制位。以下四个表达式的计算结果是?

		\begin{lstlisting}[language=c, numbers=none]
(signed char)(114)
(signed char)(514)
(signed char)(1145 + 14)
(signed char)(114514)
		\end{lstlisting}

		\item 向该程序输入以下数字,输出结果分别是多少?

		\begin{lstlisting}[numbers=none]
256
		\end{lstlisting}
		\begin{lstlisting}[numbers=none]
128
		\end{lstlisting}
		\begin{lstlisting}[numbers=none]
114
		\end{lstlisting}
		\begin{lstlisting}[numbers=none]
514
		\end{lstlisting}

		提示:先利用上一小问求出 \lstinline{x} 的值,然后计算表达式 \lstinline{x & (-x)},最后利用上一小问求得输出。

		\item 该程序的输出的含义是什么?

		提示:利用二进制进行分析。
	\end{enumerate}

	\item 浮点数的运算结果与运算速度。将以下代码输入到你的代码编辑器中。

	\begin{lstlisting}[language=c, moreemph={[1]clock_t}, moreemph={[2]clock}, moreemph={[3]CLOCKS_PER_SEC}]
#include <stdio.h>
#include <time.h>

int main()
{
	// 输入部分 开始
	// 这道题没有输入
	// 输入部分 结束
	clock_t start_time = clock();
	// 运算部分 开始

	// 运算部分 结束
	printf("%.3f seconds elapsed\n",
		(double)(clock() - start_time) / CLOCKS_PER_SEC);
}
	\end{lstlisting}

	提示:你可以不输入代码中的注释。

	\begin{enumerate}
		\item 该程序的第 14 行有一个表达式。仿照它写出以下表达式:

		\begin{lstlisting}[language=c, numbers=none]
(double)1500L / 1000L
		\end{lstlisting}

		回答下列问题:
		\begin{enumerate}
			\item 该表达式中,\lstinline{1500L} 里的 \lstinline{L} 的含义或作用是什么?
			\item 该表达式的计算结果是什么?
			\item 该表达式中,\lstinline[language=c]{(double)} 的直接作用是什么?将它去掉后,表达式的计算结果又是多少?
		\end{enumerate}

		\item 该程序的作用是对运算部分的代码进行计时。在运算部分键入以下代码:

		\begin{lstlisting}[language=c, firstnumber=11]
	float number = 0.0f;
	for (int i = 0; i < 1000000000; i++)
	{
		float temp = 0.1f;
		temp *= 0.1f;
		number += temp;
	}
	printf("%f\n", number);
		\end{lstlisting}

		\textbf{在 Debug 配置下}运行程序,程序的输出是什么?

		\item 在运算部分键入以下代码:

		\begin{lstlisting}[language=c, firstnumber=11]
	double number = 0.0;
	for (int i = 0; i < 1000000000; i++)
	{
		double temp = 0.1;
		temp *= 0.1;
		number += temp;
	}
	printf("%f\n", number);
		\end{lstlisting}

		\textbf{在 Debug 配置下}运行程序,程序的输出是什么?

		\item 对以上两小问的程序输出进行解释。对比以上两问程序的运行时间,你能得出什么结论?

		提示:人类常把圆周率 $\pi$ 近似为 $3.14$,而很少把 $\pi$ 近似为 $3.14159265358979323846264$,原因是后者不适合人类进行计算,而前者的精度已经满足了人类在部分场景下的需求。
		% 参考:https://blog.csdn.net/u010839382/article/details/52775863
	\end{enumerate}

	\item \lstinline[language=c]{switch-case} 语句。将以下代码输入到你的代码编辑器中。

	\begin{lstlisting}[language=c]
#include <stdio.h>

int main()
{
	int clause;
	scanf("%d", &clause);
	switch (clause)
	{
	case 0:
		printf("Zero\n");
		break;
	case 2:
		printf("One\n");
	case 1:
		printf("One\n");
		break;
	case 3:
		printf("Three\n");
	default:
		printf("Others\n");
		break;
	}
}
	\end{lstlisting}

	\lstinline[language=c]{switch-case} 语句是 C 语言支持的一种\textbf{分支结构},完成以下任务。

	\begin{enumerate}
		\item 向该程序输入以下数字,输出结果分别是多少?对于每一个输入,调试程序,从第 7 行开始记录之后完整的当前行变化情况。
		\begin{lstlisting}[numbers=none]
0
		\end{lstlisting}
		\begin{lstlisting}[numbers=none]
1
		\end{lstlisting}
		\begin{lstlisting}[numbers=none]
2
		\end{lstlisting}
		\begin{lstlisting}[numbers=none]
3
		\end{lstlisting}
		\begin{lstlisting}[numbers=none]
4
		\end{lstlisting}

		提示:为了便于理解题意,下面给出输入为 \lstinline{4} 时的答案,所以\textbf{你只需要给出输入为 \lstinline{0, 1, 2, 3} 时的答案。}但你应该首先检查输入为 \lstinline{4} 时情况是否与以下答案符合。

		\begin{enumerate}
			\item 直接运行程序,输入 \lstinline{4},得到程序的输出结果为 \lstinline{Others}(可以忽略行末空格与文末回车)。
			\item 在第 7 行下断点,然后点击调试程序,输入 \lstinline{4},程序暂停,当前行为第 7 行。
			\item 点击逐语句,当前行为第 20 行。
			\item 点击逐语句,当前行为第 23 行。程序运行结束。
		\end{enumerate}

		\item 根据第一小问的答案说明 \lstinline[language=c]{switch-case} 语句中 \lstinline[language=c]{case} 和 \lstinline[language=c]{break} 语句的作用。

		\item 该程序有一处 \lstinline[language=c]{break} 语句是多余的,指出它,并说明它多余的原因。

		\item 假设该程序想要实现的效果是:输入一个整数,若这个整数为 \lstinline{0, 1, 2, 3},则输出一些英文单词,这些英文单词对应的数的和等于输入的数;若这个整数为其他值,则输出 \lstinline{Others}。根据第一小问的答案,可以发现程序没有按预期工作。添加一个 \lstinline[language=c]{break} 语句即可修正该错误,指出应当在哪里添加这个 \lstinline[language=c]{break} 语句。

		\item 不使用 \lstinline[language=c]{switch-case} 语句,改写该程序,使得程序的运行结果在任何输入下都保持不变(与经过上一小问的修正后的程序相比)。
	\end{enumerate}

	\item 自增自减运算符。C 语言创新地提出了以下四个运算符:
	\begin{itemize}
		\item 前置自增运算符(\lstinline{++A})。语句 \lstinline[language=c, moreemph={[2]f}]{f(++x);} 等价于:

		\begin{lstlisting}[language=c, numbers=none, moreemph={[2]f}]
x += 1;
f(x);
		\end{lstlisting}

		\item 后置自增运算符(\lstinline{A++})。语句 \lstinline[language=c, moreemph={[2]f}]{f(x++);} 等价于:

		\begin{lstlisting}[language=c, numbers=none, moreemph={[2]f}]
f(x);
x += 1;
		\end{lstlisting}

		\item 前置自减运算符(\lstinline{--A})。运算规则类似于前置自增运算符。

		\item 后置自减运算符(\lstinline{A--})。运算规则类似于后置自增运算符。
	\end{itemize}

	它们统称为\textbf{自增自减运算符},因为能简化代码而在 C 语言中得到广泛运用。完成以下任务。

	\begin{enumerate}
		\item 若 \lstinline{x} 是 C 语言中的一个变量,初值为 \lstinline{0},则表达式 \lstinline{x++} 的值是多少?表达式 \lstinline{++x} 的值又是多少?

		提示:考虑定义 \lstinline[language=c, moreemph={[2]f}]{f} 为:

		\begin{lstlisting}[language=c, numbers=none, moreemph={[2]f}]
void f(int number)
{
	printf("%d\n", number);
}
		\end{lstlisting}

		\item 含有自增自减运算符的表达式一定具有副作用,计算表达式时一定会改变程序中某些变量的值,\textbf{如果将多个有副作用的表达式混在一起进行计算,则可能产生程序员难以预料的结果}。运行以下程序,记录输出结果。

		\begin{lstlisting}[language=c]
#include <stdio.h>

int main()
{
	int x;
	x = 114;
	x = ++x + x;
	printf("%d\n", x);

	x = 114;
	x = x++ + x--;
	printf("%d\n", x);
}
		\end{lstlisting}

		提示:只需要记录,不需要解释。

		\item 网络上相传,C 语言有一个“趋向于运算符”(\lstinline{-->})。

		\begin{lstlisting}[language=c]
#include <stdio.h>

int main()
{
	int x = 5;
	while (x --> 0)
	{
		printf("%d\n", x);
	}
}
		\end{lstlisting}

		事实上,所谓“趋向于运算符”是两个运算符:后置自减运算符、大于运算符。不使用自减运算符,写出以上程序的等价程序,使得程序的运行结果保持不变。
	\end{enumerate}

	\item 时间复杂度。首先阅读下面的代码。

	\begin{lstlisting}[language=c]
#include <stdio.h>

int main()
{
	int n;
	scanf("%d", &n);
	int counter = 0;
	{ // 计数 开始
		counter++;
	} // 计数 结束
	printf("%d\n", counter);
}
	\end{lstlisting}

	该程序执行的指令数与输入 \lstinline{n} 无关,所以这个程序的(渐进)时间复杂度可以记为 $O(1)$。

	\begin{enumerate}
		\item 将该程序的计数部分(第 9 行)替换为以下代码。

		\begin{lstlisting}[language=c, firstnumber=9]
		for (int i = 1; i <= n; i++)
			counter++;
		\end{lstlisting}

		则程序执行的指令数可以被变量 \lstinline{counter} 的值的有限倍数控制,而 \lstinline{counter} 的值与输入 \lstinline{n} 有关。程序的时间复杂度是多少?请使用大 $O$ 记号表示,但请尽可能地接近上确界。规定自变量为 $n$。

		\item 将该程序的计数部分替换为以下代码。

		\begin{lstlisting}[language=c, firstnumber=9]
		for (int i = 1; i <= n; i++)
			for (int j = 1; j <= i; j++)
				counter++;
		\end{lstlisting}

		则程序执行的指令数可以被变量 \lstinline{counter} 的值的有限倍数控制。问题同上一小问。

		\item 将该程序的计数部分替换为以下代码。

		\begin{lstlisting}[language=c, firstnumber=9]
		for (int i = 1; i <= n; i++)
			for (int j = 1; j <= n; j += i)
				counter++;
		\end{lstlisting}

		运行程序,并使用计算器计算,填写下面的表格(已帮你填写了第一列)。

		\begin{table}[H]
			\centering
			\begin{tabular}{|c|c|c|c|c|c|c|}
				\hline
				$n$ & 100 & 1000 & 10000 & 100000 & 1000000 & 10000000
				\\\hline
				\lstinline$counter$ & 573 & \phantom{8053} & \phantom{103643} & \phantom{1266714} & \phantom{14969985} & \phantom{172725300}
				\\\hline
				$n \ln n$ & 460.5 & \phantom{6907.8} & \phantom{92103.4} & \phantom{1151292.5} & \phantom{13815510.6} & \phantom{161180956.5}
				\\\hline
				$\phantom{\Biggl|} \dfrac{\text{\lstinline$counter$}}{n \ln n} \phantom{\Biggr|}$ & 1.244 & \phantom{1.166} & \phantom{1.125} & \phantom{1.100} & \phantom{1.084} & \phantom{1.072}
				\\\hline
			\end{tabular}
		\end{table}

		根据以上表格,猜测程序的时间复杂度是多少?请使用大 $O$ 记号表示,但请尽可能地接近上确界。规定自变量为 $n$。
	\end{enumerate}

	\item 在线评测平台题单。

	\begin{enumerate}
		\item 在在线评测平台(OJ)上完成下面的所有题目,得到 Accepted(AC)的反馈。其中的“自选”表示你自己感兴趣的其他题目,也必须完成。自选的题目不限制 OJ 平台。
		\item 从下面的题目中选择 2 道(包括自选)撰写题解。题解应当包含题意解析、你的分析思路、你遇到的问题、你的标准解法。
	\end{enumerate}

	\begin{table}[H]
		\centering
		\begin{tabular}{|c|c|c|}
			\hline
			题号 & 题目名称 & 备注
			\\\hline
			\href{https://www.luogu.com.cn/problem/P5710}{洛谷 P5710} & 数的性质 & 逻辑运算符
			\\\hline
			\href{https://www.luogu.com.cn/problem/P1085}{洛谷 P1085} & 不高兴的津津 & 使用循环结构
			\\\hline
			\href{https://www.luogu.com.cn/problem/P1909}{洛谷 P1909} & 买铅笔 & 仔细读题,只买同一种包装的铅笔
			\\\hline
			\href{https://www.luogu.com.cn/problem/P1046}{洛谷 P1046} & 陶陶摘苹果 &
			\\\hline
			\href{https://www.luogu.com.cn/problem/P5718}{洛谷 P5718} & 找最小值 & 注意与找最大值的区别
			\\\hline
			\href{https://www.luogu.com.cn/problem/P5719}{洛谷 P5719} & 分类平均 & 输出一位小数的格式化说明符为 \lstinline[language=c]$"%.1f"$
			\\\hline
			\href{https://www.luogu.com.cn/problem/P5720}{洛谷 P5720} & 一尺之棰 &
			\\\hline
			\href{https://www.luogu.com.cn/problem/P1980}{洛谷 P1980} & 计数问题 & \lstinline[language=c]$x % 10$ 即为 \lstinline$x$ 的个位
			\\\hline
			\href{https://www.luogu.com.cn/problem/P1035}{洛谷 P1035} & 级数求和 &
			\\\hline
			\href{https://www.luogu.com.cn/problem/P2669}{洛谷 P2669} & 金币 & 牺牲时间复杂度免去公式推导,反之则可免去循环设计
			\\\hline
			\href{https://www.luogu.com.cn/problem/P1075}{洛谷 P1075} & 质因数分解 & 注意时间复杂度
			\\\hline
			自选 & &
			\\\hline
			自选 & &
			\\\hline
			自选 & &
			\\\hline
		\end{tabular}
	\end{table}

	提示:非自选题目虽然较多,但难度不高。要覆盖所有知识点请自行复习。

	提示:在线评测平台为黑盒测试,只要程序在给定输入的情况下给出正确输出就算通过。

\end{problemset}